\documentclass{article}
\usepackage[utf8]{inputenc}
\usepackage[margin=0.5in]{geometry}
\usepackage{graphicx}
\usepackage{amsmath}
\graphicspath{ {../figs/} }


\title{Problem Set 3: Quantitative Economics (ECON 8185-002)}
\author{Teerat Wongrattanapiboon}
\date{30 November 2021}

\begin{document}

	\maketitle
	
	\noindent\textbf{\Large Exogenous labor supply} \\
	
	Assume utility function, its first derivative, and the inverse of the first derivative are 
	
	$$U(c) = \frac{c^{1-\gamma}}{1-\gamma},$$
	$$Uc(c) = c^{-\gamma},$$
	$$Uc^{-1}(x) = x^{-\frac{1}{\gamma}}.$$
	
	Household's budget constraint and the skill process are
	
	$$c_{t} + a_{t+1} \leq e_{i}w + (1+r)a_{t}.$$
	$$a_{t+1} \geq \underline{a} = 0.$$
	
	The skill process is
	
	$$\log(e_{i,t}) = \rho\log(e_{i,t-1}) + \sigma \epsilon_{i,t}.$$
	
	The production function is 
	
	$$Y_{t} = F(K{t},N_{t}) = AK_{t}^{\theta}N_{t}^{1-\theta}$$
	
	It is useful to set $A$ such that steady state $Y = 1$. We will use endogenous grid method to solve for stationary equilibrium for this economy. The algorithm is as follows: \\ 
	
	1) Guess r \\
		
	2) Solve for w with \\
	
	$$w(r) = (1-\theta)\left(\frac{r+\delta}{\theta}\right)^{\frac{\theta}{\theta-1}}. $$
	
	3) Given a grid for $(a,\epsilon)$, we first guess $c^{j}(a,\epsilon) = ra + w\epsilon$. \\
	
	4) Then, for all $(a_{k}',\epsilon_{j})$, we substitute a guess for the consumption tomorrow, $c^{j}(a_{k}',\epsilon_{j})$, in the Euler equation to solve for current consumption
		
		$$\bar{c}(a_{k}', \epsilon_{j}) = U_{c}^{-1}\left[ \beta (1+r) \sum_{\epsilon'}P(\epsilon'|\epsilon_{j}) \cdot U_{c}\left[ c^{j}(a_{k}',\epsilon') \right] \right].$$
		
		We then obtain current assets given consumption today defined on asset grid tomorrow as follows:
		
		$$\bar{a}(a_{k}', \epsilon_{j}) = \frac{\bar{c}(a_{k}', \epsilon_{j}) + a_{k}' - w\epsilon_{j}}{1+r}.$$
		
	5) Then we update $c^{j+1}(a,\epsilon)$ as follows:
		
		$$c^{j+1}(a,\epsilon) = (1+r)a + w \epsilon  \text{ for all } a \leq \bar{a}(a_{1}', \epsilon)$$
		$$c^{j+1}(a,\epsilon) = \text{ Interpolate } [\bar{c}(a_{k}', \epsilon), \bar{c}(a_{k+1}', \epsilon)] \text{ when } a \in [\bar{a}(a_{k}', \epsilon), \bar{a}(a_{k+1}', \epsilon)]$$
		
		Note here that $(a_{k}', \epsilon_{j})$ are on the same grid as $(a,\epsilon)$. The first case is when the borrowing constraint is binding.  \\
		
	6) Repeat step 4 and 5 until $c^{j}(a,\epsilon)$ converged. Observe that we no longer need a root finding procedure, but still need to interpolate the optimal policy on our defined grids in this EGM algorithm. \\
		
	 
	
	\noindent\textbf{\Large Endogenous labor supply} \\
	
		We now incorporate endogenous labor supply in our utility function as follows:
		$$U(c,l) =  \frac{c^{1-\gamma}}{1-\gamma} - \phi\frac{l^{1+\eta}}{1+\eta}$$
		$$Ul(l) = \phi l^\eta$$
		$$Ul^{-1}(x) = (x/\phi)^{1/\eta}$$
		
		where $l$ represents labor. The budget constraint now becomes
		
		$$c+a' \leq \epsilon w l + (1 + r)a.$$
		
		To implement endogenous grid method, as in the case of exogenous labor supply, we start by \\ 
		
		1) Guess r \\
		
		2) Solve for w with \\
		
		$$w(r) = A(1-\theta)\left(\frac{r+\delta}{A\theta}\right)^{\frac{\theta}{\theta-1}}. $$
		
		3) Then we guess $c^{j}(a,\epsilon) = ra + w\epsilon$ as before. \\
		
		4) Use this guess to solve for $\bar{l}(a, \epsilon)$ from
		
		$$\frac{u_{l}(c^{j}(a,\epsilon),l)}{u_{c}(c^{j}(a,\epsilon),l)} = \frac{(c^\eta l^{1-\eta})^{-\mu}c^{\eta}(1-\eta)l^{-\eta} }{(c^\eta l^{1-\eta})^{-\mu}l^{1-\eta}(\eta) c^{\eta-1}} = (\frac{1-\eta}{\eta}) \cdot \frac{c}{l} = w\epsilon.$$
		
		Note that above is an equation of only one unknown, $l$, given a guess for $c$, $c^{j}(a,\epsilon)$. \\
		
		
		5) Then, for all $(a_{k}',\epsilon_{j})$, we use $\bar{l}(a_{k}',\epsilon_{j})$ and $c^{j}(a_{k}',\epsilon_{j})$ to solve for 
		
		$$\bar{c}(a_{k}', \epsilon_{j}) = U_{c}^{-1}\left[ \beta (1+r) \sum_{\epsilon'}P(\epsilon'|\epsilon_{j}) \cdot U_{c}\left[ c^{j}(a_{k}',\epsilon'),\bar{l}(a_{k}',\epsilon') \right] \right]$$
		
		$$\bar{a}(a_{k}', \epsilon_{j}) = \frac{\bar{c}(a_{k}', \epsilon_{j}) + a_{k}' - w\epsilon_{j}(1-\bar{l}(a_{k}',\epsilon_{j}))}{1+r}.$$
		
		Note here that $(a_{k}', \epsilon_{j})$ are on the same grid as $(a,\epsilon)$. \\
		
		6) Then we update $c^{j+1}(a,\epsilon)$ as follows:
		
		$$c^{j+1}(a,\epsilon) = (1+r)a + w \epsilon (1- \bar{l}(a,\epsilon)) \text{ for all } a \leq \bar{a}(a_{1}', \epsilon)$$
		$$c^{j+1}(a,\epsilon) = \text{ Interpolate } [\bar{c}(a_{k}', \epsilon), \bar{c}(a_{k+1}', \epsilon)] \text{ when } a \in [\bar{a}(a_{k}', \epsilon), \bar{a}(a_{k+1}', \epsilon)]$$
		
		Then we solve for stationary distribution $\lambda$. With this, we then solve for equilibrium $r$. \\
		
		1) Guess $r_{0}$. \\
		
		2) Solve for policy functions $a'(a,\epsilon),c(a,\epsilon)$ and $l(a,\epsilon)$. \\
		
		3) Use $a'(a,\epsilon)$ to compute $\lambda(a,\epsilon)$. \\
		
		4) Compute aggregate supply of capital (savings) $K_{s}(r_{0}) = \int a_{i}di$ and aggregate labor supply $N(r_{0}) = \int \epsilon_{i}l_{i}di$. \\
		
		5) Compute $r_{s} = A\theta \left( \frac{K_{s}(r_{0})}{N(r_{0})} \right)^{\theta-1} -\delta$. This is an interest rate as implied by aggregate supply of capital. \\
		
		6) Compare $r_{0}$ and $r_{s}$. If they are close enough, then we solve for an equilibrium interest rate $r = \frac{r_{0}+r_{s}}{2}$. Otherwise, set 
		
		$$r_{0} = 0.8 \cdot r_{0} + 0.2 \cdot r_{s}$$
		
		and repeat from (1) through (6) until $r$ converged.
	
	
	
\end{document}